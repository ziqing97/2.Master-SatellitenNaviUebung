\documentclass[%
paper=a4,      % alle weiteren Papierformat einstellbar
fontsize=11pt, % Schriftgr��e (12pt, 11pt (Standard))
BCOR1cm,       % Bindekorrektur, bspw. 1 cm
DIV15,         % f�hrt die Satzspiegelberechnung neu aus s. scrguide 2.4
%twoside,       % Doppelseiten
headsepline,   %
headings=openright, % Kapitel nur rechts beginnen
%biblography=totoc, % Literaturverzeichnis einf�gen bibtotocnumbered: nummeriert
parskip=half,  % Europ�ischer Satz mit Abstand zwischen Abs�tzen
chapterprefix, % Kapitel anschreiben als Kapitel
headsepline,   % Linie nach Kopfzeile
titlepage,     %
numbers=noenddot,
%draft	       % zeigt �berlange Zeilen an
]{scrreprt}

\usepackage{pdfpages}       % Titelseite hat ein anderes Layout. Sie wird 
                            % separat erzeugt und hier eingef�gt
\usepackage[T1]{fontenc}
\usepackage[utf8]{inputenc}  % Zeichencodierung
\usepackage[ngerman, english]{babel} % Worttrennung nach neuer Rechtschreibung
%\usepackage[ngerman]{babel}
\usepackage{siunitx}
\usepackage{ellipsis}       % Leerraum um Auslassungspunkte
\usepackage{fixltx2e}       % Fehlerkorrektur Zeichens�tze
\usepackage{xspace}         % f�ge evtl. notwendiges Leerzeichen hinzu (\xspace)
\usepackage{textcomp}
\usepackage{bm}
\usepackage{subfigure}

%\usepackage{mathptmx}           % Times + passende Mathefonts
\usepackage{mathpazo}           % Palatino + passende Mathefonts
\usepackage[scaled=.92]{helvet} % skalierte Helvetica als \sfdefault
\usepackage{courier}            % Courier als \ttdefault

\usepackage{graphicx}    % Einbindung von Grafiken
\graphicspath{{Images/}} % Unterverzeichnis, in dem Grafiken abgelegt werden
\usepackage{listings}    % Listenausgabe externer Dateien

\usepackage{float}      % Paket zum Erweitern der Floatumgebungen
\usepackage[figuresright]{rotating}   % Rotieren von Objekten
%\usepackage{hvfloat}
\usepackage{array}      % Paket zum Erweitern der Tabelleneigenschaften
\usepackage{booktabs}   % Paket f�r sch�nere Tabellen

\usepackage{amsmath}    % erweiterte Mathematik-Umgebungen
\usepackage{amssymb}
\usepackage{url}

\usepackage{framed}
\usepackage{color}
\definecolor{bk}{rgb}{0.92,0.92,0.92}


% Einstellungen f�r das Literaturverzeichnis
\usepackage[round]{natbib}
\setlength{\bibsep}{0.5\baselineskip}
\setlength{\bibhang}{1cm}
\bibliographystyle{agsm}

% Andere Schriftarten in Koma-Script
\setkomafont{sectioning}{\normalfont\bfseries}
\setkomafont{captionlabel}{\rmfamily\bfseries\small}
\setkomafont{caption}{\mdseries\itshape\small}
\setkomafont{pagehead}{\normalfont\itshape} % Kopfzeilenschrift
\setkomafont{descriptionlabel}{\normalfont\bfseries}

% Kopf und Fu�zeilen
\usepackage[automark]{scrlayer-scrpage}

% Hyperref
\usepackage{hyperref}

% Literaturverzeichnis-Stil
\bibliographystyle{plain}

% weitere Einstellungen
\tolerance=200               % �bervolle Zeile vermeiden
\emergencystretch=3em

\clubpenalty=10000           % 'Schusterjungen' und 'Hurenkinder' vermeiden
\widowpenalty=10000 
\displaywidowpenalty=10000

\parindent 0pt               % Einzug zu Absatzbeginn festlegen

\setcapindent{1em}           % Zeilenumbruch bei Bildbeschreibungen.

\setcounter{secnumdepth}{3}  % Strukturiertiefe bis subsubsection{} m�glich
\setcounter{tocdepth}{3}     % Dargestellte Strukturiertiefe im Inhaltsverzeichnis

% Korrekturversion mit 1.5-fachem Zeilenabstand im Hauptteil:
\newif\ifiscorrect
%\iscorrecttrue   % Korrekturversion
\iscorrectfalse % keine Korrekturversion

%% Eigene Definitionen:

% Einheiten:
\def\ut#1{\ensuremath{\,\mathrm{#1}}}

% Operatoren:
\def\grad{\ensuremath{\mathop{\mathrm{grad}}\nolimits}}
\def\transp#1{\ensuremath{{#1}^\mathsf{T}}}  % transpose
\def\const{\ensuremath{\mathop{\mathrm{const.}}\nolimits}}

% Formelzeichen:
\def\vec#1{\ensuremath{\mathbf{#1}}}
\def\matr#1{\ensuremath{\mathbf{#1}}}

% Hack, um ein zus�tzliches Leerzeichen nach \input zu entfernen:
\def\myinput#1{%
  \endlinechar=-1 % kein Zeilenabschlusszeichen
  \input #1\relax
  \endlinechar `\^^M % Zeilenabschluss = Zeilenvorschub
}

